\section{Context}
Nowadays developers have access of a vast pool of computational resources to
that support several resource-consuming tasks, such as real-time load balancing, molecular
simulations with fairly accurate results, among other. At the same time, computer
scientists struggle with hardware limitations, more specifically, transistor size.

With the current advances in transistor technology, the design of 3D layered silicon 
chips and 5nm lithography, Moore's Law is now a problem because it is more diffcult to push the 
limits of -for single chips- at the same rate as ten years ago\cite{7478302,EETIMES1}.
The Altera Corporation stated in 2007 in a technical report that: \emph{``For most of 
the microprocessor's history, application demands have risen in response to 
processor improvements, allowing processors to stay ahead of demand. In the last 
few years, however, the situation has changed. High-performance computing (HPC) 
applications are now demanding more than processors alone can deliver, creating a 
technology gap between demand and performance.''}\cite{ALTERA_Accel_fpga}

The use of heterogeneous computer architectures is one of the most popular methods
nowadays to tackle this so called \emph{Technology Gap Problem}. Solutions like Nvidia CUDA
\cite{NVIDIA_volta} allows developers to harness the massive parallel power of present-day
GPUs to perform intensive tasks, as long as the problem fits the constraints imposed
by the hardware. 

As such, there is a growing interest on \emph{high-performance and low-power custom computing 
machines} which aim to design \emph{Application-specific integrated circuits} (ASICs) in order 
to solve  specific computational problems. But as ASIC design, prototyping, and implementation is
very costly, \emph{Field Programmable Gate Arrays} (FPGAs) have proven to be a cost-effective 
solution to implement those designs, and interconnect them with our current platforms to form 
reconfigurable computing architectures \cite{ReconfigurableComputing, 6589302,
Compton00anintroduction}. This architectures are limited only by their power-budget while offering a scalable
compute model, while at the same time overcoming several of GPGPU current problems including energy consumption
and performance degradation when facing intensive branch-divergence control-flow algorithms 
\cite{6376229}. This makes FPGA a perfect target platform for applications in robotics but
also on high-performance computing. 

\subsection{Motivation}
On non-traditional databases, metric spaces indices offer an abstraction to classify objects using
the underlying notion of space and distance which is naturally for humans to understand. Those indices 
work by placing elements on a hiperdimensional grid, along with a distance function which
allows to measure the distance between any two objects in the dataset.
One of the many approaches to perform similarity searches is to perform k-nn or range queries 
on permutant-based indices, which abstract the dataset dimensionality and the cost of the 
object-object distance calculation. To perform a search on this index, a permutation is generated
for the query object and then compared to the whole dataset under the premise that computing
distance between two permutations is faster than computing the full distance between the two
elements. After their distances are computed, a subset is selected under a certain 
criteria given by the query nature and the results are filtered later in order to answer the query.
\cite{5271946,5271944}

The implementation of indices for approximate search on metric spaces pose a great problem, given that although
the operations involved in the distance, calculations are rather easy to perform on a CPU, they
are executed individually for each element of the index, regardless of them being a part of the
solution or not. This behaviour makes similarity search indices the perfect test bench for its 
implementation on a reconfigurable computing device\cite{5999889} such as the Adapteva Parallella-16,
a \emph{System on chip} (SoC) based on the Xilinx All Programmable Zynq7000 Series SoC which packs together
a Dual-core 32-bit ARM Cortex-A9 host controller and a Artix-7 FPGA \cite{DBLP:journals/corr/OlofssonNZ14,LINLEY_1}.
The board also has an Epiphany III multicore accelerator coprocessor. Interconnected with the host-controller
though an FPGA designed ASIC in order to be used as a low-power high-performance heterogeneous computing 
platform.

One of the main problems with reconfigurable computing is the complexity of circuit design \cite{SPA_thesis}. 
Independently of the algorithm being ported to an FPGA implementation, there is no automated way to use
the same source code used on the Processing System (PS) version of the problem in the Programmable Logic (PL). 
As there are some tools developed for both FPGA manufacturers and 3rd parties to design circuits using the C 
language, they lack precision, and many considerations and ``compile'' optimizations must be performed 
in order to successfully port certain algorithms\cite{SPA_thesis}, such as instruction pipelines, read/write 
syncronization, clock gating\cite{XILINX_clockgating}, etc \cite{XILINX_axi,XILINX_clockgating,
XILINX_ddr_rate,XILINX_mem_interface}. As such, this work intends to serve as a guide to future hardware
designers and developers and to demonstrate that with proper study we can port some algorithms to
certain reconfigurable computing machines without major hassle.

\subsection{Goals}
\paragraph{Main objective}
\begin{itemize}
\item Study the feasibility of implementing hardware-based accelerators for the Adapteva Parallella SoC by implementing
hardware-accelerated indices for metric space indexing.
\end{itemize}

\paragraph{Specific objectives}
% \emph{(Una lista de puntos que detallan el objetivo general.)}
\begin{itemize}
    \item Specify requirements and considerations to be accounted when porting general purpose algorithms to FPGAs.
    \item Study and implement a data sharing solution between the Programmable Logic and the Programmable Software.
    \item Develop a functional FPGA-based hardware-accelerator prototype for a subset of routines involved 
    on approximate similarity search.
    \item Deliver a solid guide to serve as a starting point to future computer scientists with little or no 
    knowledge about hardware design. 
\end{itemize}