\section{Field Programmable Gate Arrays}
\subsection{Overview}
Field Programmable Gate Arrays (or FPGA for short) are programmable devices that contains logical blocks arranged as a grid that can be configured and
interconnected as will in a \emph{in-situ} way. The programable feature of this chips enables the replication of simple logic circuits sucj as logical
gates to complex designs such as \emph{Digital Signal Processors}. Commonly these chips are used in \emph{Application-Specific Integrated Circuits}
prototyping because the process behind its programming is similar to \emph{waffer synthesis} used on the creation of ASICs and their design flow
flexibility, but their energy consumption, thermal disipation and performance are worse than a dedicated circuit. The internal resources of an FPGA 
chip consist of a matrix of configurable logic blocks (CLBs) surrounded by a periphery of I/O blocks. Signals are routed within the FPGA matrix by 
programmable interconnect switches and wire routes. Because of the latter point, those chips are commonly used on robotics and hardware development.

\subsection{FPGA Programming and its difficulties}
Most programming on FPGA is done using \emph{hardware description languages} such as \emph{VHDL} and \emph{VERILOG} which describes the circuit to be
loaded onto the FPGA using a \emph{register-transfer level} abstraction of synchronous digital circuit models a in terms of the flow of digital signals 
between each one of the hardware registeres and the logical operations performed on those signals as well. This generates a high-level representation
of a circuit, from which lower-level representations and ultimately actual wiring can be derived. 

As such, a single FPGA can replace thousands of discrete components by incorporating millions of logic gates in a single integrated circuit (IC) chip. 
Given that FPGAs are huge fields of programable gates, its programming allows the creation of multiple hardware paths, delivering a trully parallel nature
in which different processing operations do not compete each other for resources (as opposite in modern CPUs in which all programs running on the OS are
competting for the available resources). At the same time, hardware execution of the problem provide more performance than most processor-based solutions
as well as a higher throughput of data, which becomes the main goal in FPGA programming.

FPGAs give software developers a promise of a greater performance than processor-based solutions, but it comes of a cost that all operations must be
implemented as it were directly on hardware, thus, for most computer engineers there is a wide knowledge gap when it comes to use it for actual data
processing.

\subsection{High-level synthesis}
In order to turn the hardware description into code, a process known as \emph{logical synthesis} is performed, which takes an abstraction of the circuit 
behavour as RTL and then turns it into a logic implementation in terms of logic gates and blocks. Then, a \emph{bitstream} is generated which is transfered
to the FPGA to reconfigure its grid.



\subsection{Interconnection between Programable Logic and Processing Systems}