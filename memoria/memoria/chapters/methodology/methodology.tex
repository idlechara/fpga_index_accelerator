\section{Methodology}
This problem required now only knowledge in computer science but also a prior background on electronics, as such, is needed to 
research about the following topics before going into serious development:

\begin{itemize}
    \item {\textbf{Non-traditional databases and metric spaces}}
    \item {\textbf{FPGA development}}
    \item {\textbf{High-level synthesis}}
    \item {\textbf{Interconnection between programable logic and program software}}
\end{itemize}
%according to the required knowledge we create a solution, why so much prior knowledge, and which elements elements do I use
\section{Implementation Methodology}
%explain why I'm using or not a certain methodology
To develop both the experiments and the acelerator we will follow an interative development cycle, based on the results of each
source code but not following a exact construction metodology because of the nature of this work. Namely the elements to iterate 
can be summarized as follows:

\begin{itemize}
    \item {\textbf{Permutant-based index testbench:} This implementation is intented to be used to gain insight of which methods are the most frequently
    executed and to use it as reference to compare the results with the acelerator-improved version. It will be developed using C++ and it will provide 
    a graphical interface to visually debug the queries.}
    \item {\textbf{FPGA Accelerator:} This artifact it will be used to test the interconnection capabilities of hardware accelerators over the Zynq board,
    and it will not provide drivers for userspace memory management rather it will operate directly with each DRAM block present on the board. It will be 
    developed using Xilinx Vivado SDK for FPGA programming and with Xilinx Vivado HLS to write C/C++ code targeted to be synthetized on the on the board
    as an acelerator IP.}
    \item {\textbf{Accelerated index testbench:} Once the FPGA accelerator is ready, this artifact will provide insight on how the accelerator implementation affects
    the performance of the original index. This implmentation will be optimized to run on ARM architectures and it will be a memory-coalesent implementation
    of the first index implemented at the beggining.}
    \item {\textbf{Kernel drivers:} This artifact will enable our accelerator to peform memory-safe operations by restricting memory access on userspace.
    It will be developed using C and designed aroung GNU Linux guidelines for kernel drivers and modules.}
    \item {\textbf{Accelerated index:} This artifact will wrap everything together, both the accelerated and non accelerated versions and it will be used
    to measure the final performance of the index.}
\end{itemize}

As it can be seen, there is a need to iterate though every artifact of this research.

\subsection{Design and architecture}
%self-explanatory, only specs not details
\subsection{Development}

\section{Testing and experimentation Methodology}
For experimentation we will follow an experimental approach rather than a purely theoretical one \cite{citation_required} to devise how the 
accelerator affects the behaivour of the index. For such purposes a series of tests will be performed:

\begin{itemize}
    \item {\textbf{Hot zones on permutant index:} The bigger benefits of implementing a stream-like accelerator are reaped when the operation
    to perform is simple and it can be streamlined and paralelized. So, after implementing the index, we want to map all code zones which are in most 
    demand and then classify by their complexity and size. This two parameters will be the base to decide which ones are feasible to implementation
    on the FPGA as accelerators.}
    \item {\textbf{Litography and synthesis:} As we will be using high-level synthesis to implement the accelerators, there is a need to study the 
    effects of different \texttt{PRAGMAS} in the accelerator source code. It is expected that pragmas that increase the thoughput of our 
    accelerator will also be in higher demmand of resources which could be not be satisfied by our entry-level development board, which also affects
    the scope of the accelerator covered as well as the synthesis time required to create a given IP.}
    \item {\textbf{Kernel Userspace I/O Drivers:} There are several ways to build kernel drivers and modules, some of them treat as mere files, while
    another approaches treat each device descriptor as text files, streams or raw memory mapping. Each one of those will have pros and cons that
    will be need to be studied in order to correct the implementation to be memory-safe and scalable.}
    \item {\textbf{Memory-safe implementation:} The implementation cannot go into a non-memory safe state as FPGA will have raw access to DRAM and this 
    can suppose many problems as the index being stuck or memory corruption to other processes. This involves the implementation of checked runtime checks
    along the execution which are expected to be done though a correct memory management on userspace and bound checking on hardware.}
\end{itemize}
%how the experiments were to be performed