\documentclass[11pt,letterpaper]{article}
\usepackage{pslatex}
\usepackage[english]{babel}
\usepackage[utf8]{inputenc} % Caracteres con acentos.
\usepackage{latexsym}
\usepackage{amssymb} 
\usepackage{amsmath}
\usepackage{epsfig}
\usepackage{url}

\begin{document}

\pagestyle{empty}

\title{
    Hardware-accelerated routines for similarity search on high-dimensionality metric spaces
    %Implementation of hardware-accelerated approximate search algorithms on a High-performance Low-Energy platform.\\
%\footnote{Los comentarios de este documento (texto en itálica), incluyendo esta nota a pie de página, deben ser removidos por el alumno al momento de elaborar su propuesta de proyecto.}
}
\author{
Erik Regla (Student)\\
%cualquier hueón sirve acá, lo importante es que me deje trabajar tranquilo y no me estorbe.
Rodrigo Paredes (Advisor)\\
Ingeniería Civil en Computación\\ 
Universidad de Talca}
\date{\today}
\maketitle


\section{Proposal description}
% \emph{
%     %(Esta sección debe incluir una presentación general del problema a investigar y/o idea a desarrollar. En esta sección se debe incluir aquellas referencias bibliográficas vinculadas al contexto del proyecto. Para esto último se recomienda el uso de un archivo *.bib, el cual usa el formato BibTex \cite{1} para codificar  referencias sobre libros \cite{2}, artículos en revistas científicas \cite{3}, artículos en conferencias o workshops \cite{4}, reportes técnicos \cite{5}, capítulos en libros \cite{6}, y páginas Web \cite{7}. La longitud máxima de esta sección es de 2.5 páginas.)
%     % We plan to study the feasibility of FPGA-based hardware accelerators for approximate search on metric spaces by implementing then on the Zynq7000 platform provided by the Parallella board.
% }


\subsection{Project's context} 
\emph{
    Moore's law is now on decline as we're approaching to atomical scale of transistors and then to minimize the lithography of processing units will be impossible. This is a source of worry for many engineers because Moore's law statements are becoming hard to maintain with each passing day. This encourages the need of new computational models and architectures to solve some problems, for instance General Purpose GPU Computing (GPGPU) in which graphic cards are used as massively parallel processors to work with large amounts of data at the cost of a reduced, and somewhat limited instruction set.
\\
%     Another problem is energy consumption rate of current hardware, which can be very aggressive depending on the target platform, thus, leading to a rise in the computational cost. This is a problem not only for the hardware consumption itself as cooling systems are needed to maintain systems running over extended periods of time.
% \\
    There is a growing interest on High-performance and low-power custom computing machines implemented on FPGAs as they pose a flexible platform to implement custom algorithms in the form of combinatorial circuits. These solutions are only limited by their power-budget, offering a scalable compute model for certain problems even after Moore's law end.
\\
    % As we reduce the dimensionality of a dataset, similarity queries degrade their accurracy. In the case of 
    % Esta sección debe incluir el marco en el cual se presenta el problema y/o proyecto a desarrollar, incluyendo los fundamentos teóricos y/o prácticos necesarios para el desarrollo del proyecto.) Esta sección responde a la pregunta ¿Dónde surge el problema?
}

\subsection{Problem definition} 
\emph{In order to perform k-nn or range queries on permutant-based indices, first, a permutation is generated for the query object and then compared to the whole dataset under the premise that computing distance between two permutations is faster than computing the full distance between the two elements. After their distances are computed, a subset is selected under a certain criteria given by the query nature and the results are filtered later in order to answer the query.
\\
As permutations are abstractions of the intrinsic dataset dimension, as we reduce the permutation/dimensionality ratio, the results become inaccurate, on the other hand, if we raise the ratio to increase acurracy we end computating the whole dataset rendering the approach useless as it's more time consuming than the original problem.
    % (Esta sección debe incluir la descripción del problema resolver o idea a desarrollar, y la motivación para hacerlo. Es decir, cual es la importancia, innovación, aporte, y/o beneficio para la ciencia y/o la humanidad). Esta sección responde a la pregunta ¿Cuál es el problema que voy a resolver?
}

\subsection{Current works} 
TODO
% \emph{
%     (Esta sección debe incluir los enfoques usados actualmente para resolver el problema. Esta sección debe contener referencias bibliográficas a trabajos relacionados al proyecto.) Esta sección responde a la pregunta ¿Qué se ha hecho para resolver el problema?
% }

\subsection{Proposed solution}
\emph{
    In order to tacke this problem, we propose to port fragments of the routines involved on both indexing and searching procedures to an FPGA-based hardware-accelerator, in the hopes of reducing both compute time and energy consumption by taking advantage of the nature of the combinatorial circuits which could be implemented directly on hardware.
\\
    To test our solution we will implement them on a Adapteva Parallella board, an heterogeneous parallel SoC capable of running linux which embeds together a 16-core Epiphany III processor, an ARM A9-based host controller and a Zynq7000 Series FPGA, all of this in a single credit-card form factor.
    % (Esta sección debe incluir el planteamiento y justificación de la solución y/o idea, incluyendo aspectos novedosos.) Esta sección responde a la pregunta ¿Cómo voy a resolver el problema planteado?
}

%\section{Hipótesis}
%\emph{(En esta sección se deben incluir una lista de afirmaciones o suposiciones las cuales se esperar responder con el desarrollo del proyecto. La longitud máxima de esta sección es de 1/2 página.)}
%\begin{itemize}
%\item El uso de ... puede facilitar ....
%\item El problema de .... puede estudiarse como ... 
%\item Las técnicas usadas en ... pueden ser aplicables para resolver el problema de ...
%\end{itemize}



\section{Goals}
% \emph{
%     % (En esta sección se deben especificar el objetivo general y los objetivos específicos del proyecto. Los objetivos deben reflejar lo que se espera lograr con el proyecto, evitando incluir características específicas de la solución. La longitud máxima de esta sección es de 1 página.)
%     }

\paragraph{Main goal}
% \emph{(Debe ser una sola frase que resuma lo que se quiere lograr.)} 
\begin{itemize}
\item Study the feasibility of hardware-based accelerators for the Adapteva Parallella SoC.
\end{itemize}

\paragraph{Specific goals}
% \emph{(Una lista de puntos que detallan el objetivo general.)}
\begin{itemize}
    \item Specify requirements and considerations to be accounted when porting general purpose algorithms to FPGAs.
    \item Devise and evaluate methods to share or transfer data between the ARM A9 processor and the FPGA. 
    \item Design a hardware accelerator on the FPGA
    \item Evaluate the performance impact for the given solution.
    % \item Deliver an effective guide to port algorithms to FPGAs
\end{itemize}



\section{Scope of the work}
% \emph{(En esta sección se debe incluir una lista de puntos que definen los límites del trabajo. La longitud máxima de esta sección es de 1/2 página.)}
\begin{itemize}
\item In this work we expect to develop a functional FPGA-based hardware-accelerator prototype for a subset of routines involved on approximate similarity search.
\item During this work we will not create a framework to develop new algorithms on FPGAs, but we expect to deliver a solid guide to serve as a basis to future hardware developers. Also, we won't work on optimizations for the original versions of the tested routines.
\item This work is limited only to research about FPGA hardware design, and to compare and contrast both implementations.
\end{itemize}



\section{Metodología}
% \emph{(En esta sección se deben describir y justificar los métodos que se usarán para lograr las metas propuestas. Una buena opción es que, por cada objetivo, se describan una serie de pasos o métodos que se usarán para alcanzarlo. La longitud máxima de esta sección es de 1 página.)}

\paragraph{Milestone 1:} ``Accelerator base design"
\begin{itemize}
\item Analise previously developed hardware-accelerators.
\item Research about resource sharing methods and techniques for the proposed architecture.
\item Implement a simple hardware-accelerator.
\end{itemize}

\paragraph{Milestone 2:} ``Permutant-based index"
\begin{itemize}
\item Research about approximate search methods for metric spaces.
\item Research about approximate search indices for metric spaces.
\item Research about workarounds for high-dimensionality metric space datasets.
\item Implement a permutant-based index and query algorithm.
\end{itemize}

\paragraph{Milestone 3:} ``Hardware-accelerated index implementation"
\begin{itemize}
\item Select code fragments to be analised.
\item Implement on Programable Logic the selected code fragments.
\item Benchmark solutions.
\end{itemize}

\section{Work plan}
% \emph{(En esta sección se debe definir como organizar y planificar, en términos de etapas y tiempo, las actividades a desarrollar así como los resultados a obtener. Esta sección debe incluir una Carta Gantt, la cual define fechas de inicio y término. La longitud máxima de esta sección es de 1 página, sin considerar la Carta Gantt.)}

TODO
% \paragraph{Etapa 1:} Desarrollar el objetivo 1 (inicio-término)
% \begin{itemize}
% \item Analizar ... (inicio-término)
% \item Redactar ... (inicio-término)
% \end{itemize}

% \paragraph{Etapa 2:} Desarrollar los objetivos 2 y 3 (inicio-término)
% \begin{itemize}
% \item Diseñar ... (inicio-término)
% \item Programar ... (inicio-término)
% \item Redactar ... (inicio-término)
% \end{itemize}


\bibliographystyle{plain}

\bibliography{referencias}


\end{document}