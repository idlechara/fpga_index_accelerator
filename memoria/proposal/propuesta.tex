\documentclass[11pt,letterpaper]{article}
\usepackage{pslatex}
\usepackage[english]{babel}
\usepackage[utf8]{inputenc} % Caracteres con acentos.
\usepackage{latexsym}
\usepackage{amssymb} 
\usepackage{amsmath}
\usepackage{epsfig}
\usepackage{url}

\begin{document}

\pagestyle{empty}

\title{
    FPGA-based hardware-accelerated similarity search implementation on the Adapteva Parallella-16 system on chip.
    % Experimental analysis for hardware accelerators on custom heterogeneous architectures.
    %Implementation of hardware-accelerated approximate search algorithms on a High-performance Low-Energy platform.\\
%\footnote{Los comentarios de este documento (texto en itálica), incluyendo esta nota a pie de página, deben ser removidos por el alumno al momento de elaborar su propuesta de proyecto.}
}
\author{
Erik Regla (Student)\\%Kuky
Rodrigo Paredes (Advisor)\\%Rapa
Civil Computer Engineering\\ 
University of Talca}
\date{\today}
\maketitle

\section{Proposal description}
% \emph{
%     %(Esta sección debe incluir una presentación general del problema a investigar y/o idea a desarrollar. En esta sección se debe incluir aquellas referencias bibliográficas vinculadas al contexto del proyecto. Para esto último se recomienda el uso de un archivo *.bib, el cual usa el formato BibTex \cite{1} para codificar  referencias sobre libros \cite{2}, artículos en revistas científicas \cite{3}, artículos en conferencias o workshops \cite{4}, reportes técnicos \cite{5}, capítulos en libros \cite{6}, y páginas Web \cite{7}. La longitud máxima de esta sección es de 2.5 páginas.)
%     % We plan to study the feasibility of FPGA-based hardware accelerators for approximate search on metric spaces by implementing then on the Zynq7000 platform provided by the Parallella board.
% }


\subsection{Project's context} 

    With the current advances in transistor technology, the design of 3D layered silicon chips and 5nm 
    litography, Moore's Law is now a problem as now is harder to push the hardware limits -for single chips-
    at the same rate as ten years ago\cite{7478302,EETIMES1}. As stated in an 2007 technical report made by Altera
    Corporation: \emph{``For most of the microprocessor's history, application demands have risen in response to 
    processor improvements, allowing processors to stay ahead of demand. In the last few years, however, the 
    situation has changed. High-performance computing (HPC) applications are now demanding more than processors 
    alone can deliver, creating a technology gap between demand and performance.''}\cite{ALTERA_Accel_fpga}


    This \emph{Technology gap} is nowdays tackled by the using of heterogeneous computer architectures, being
    General Purpose GPU Computing (GPGPU) one of the most recurrent solutions due to the ease of programming
    and solution design provided by hardware-specific programming languages like Nvidia CUDA
    \cite{NVIDIA_volta}. But even promising solutions like GPGPU have their shortcommings, as energy consumption and branch divergence heavily degrades performance on control-flow intensive algorithms
    -which are the most common- \cite{6376229}.


    There is a growing interest on \emph{high-performance and low-power custom computing machines} which 
    aim to design \emph{Application-specific integrated circuits} (ASICs) in order to solve certain 
    computational problems. But as ASIC design, prototyping, and implementation is very costly,
    \emph{Field Programable Gate Arrays} (FPGAs) has been proven as a cost-effective solution to implement those 
    designs and interconnect them with our current platforms to form reconfigurable computing architectures
    \cite{ReconfigurableComputing, 6589302} being only limited by their power-budget while offering 
    a scalable compute model for certain problems even after Moore's law end.

%\\
    % As we reduce the dimensionality of a dataset, similarity queries degrade their accurracy. In the case of 
    % Esta sección debe incluir el marco en el cual se presenta el problema y/o proyecto a desarrollar, incluyendo los fundamentos teóricos y/o prácticos necesarios para el desarrollo del proyecto.) Esta sección responde a la pregunta ¿Dónde surge el problema?
\subsection{Problem definition} 
    One of the many approaches to perform similarity searches is to perform k-nn or range queries on permutant-based 
    indices, which abstract the dataset dimensionality and the cost of the object-object distance calculation. To 
    perform a search on this index, a permutation is generated for the query object and then compared to the whole 
    dataset under the premise that computing distance between two permutations is faster than computing the full 
    distance between the two elements. After their distances are computed, a subset is selected under a certain 
    criteria given by the query nature and the results are filtered later in order to answer the query.\cite{5271946,5271944}


    As permutations are abstractions of the intrinsic dataset dimension, as we reduce the permutation/dimensionality 
    ratio, the results become inaccurate, on the other hand, if we raise the ratio to increase acurracy we end 
    computating the whole dataset rendering the approach useless as it's more time consuming than the original 
    problem.

    This behaivour makes similarity search indices a perfect testbench for its implementation on a reconfigurable
    computing device\cite{5999889} such as the Adapteva Parallella-16, a \emph{System on chip} (SoC) based on the 
    Xilinx All Programable Zynq7000 Series SoC which packs a Dual-core 32-bit ARM Cortex-A9 host controller and a 
    Artix-7 FPGA \cite{DBLP:journals/corr/OlofssonNZ14,LINLEY_1}.
    The board also has an Epiphany III multicore accelerator coprocesor, interconnected with the host-controller
    though an FPGA designed ASIC in order to be used as a low-power high-performance heterogeneous computing 
    platform.

    The main problem with reconfigurable computing is the complexity of circuit design\cite{SPA_thesis}. 
    Independently of the algorithm being ported to an FPGA implementation, there is not automated way to use
    the same source code used on the Processing System (PS) version of the problem in the Programable Logic (PL). 
    As there are some tools developed for both FPGA manufacturers and 3rd parties to design circuits using the C 
    language, they lack of precision and many considerations and ``compile'' optimizations must be performed 
    in order to successfully port certain algorithms\cite{SPA_thesis}, such as instruction pipelining, read/write 
    syncronization, clockgating\cite{XILINX_clockgating}, etc \cite{XILINX_axi,XILINX_clockgating,
    XILINX_ddr_rate,XILINX_mem_interface}.

    % On top of that, newcomers to reconfigurable computing architectures have to deal with hardware imposed
    % constraints like a resticted component availability depending on the target device and in some cases,
    % the implementation of interconnect mechanisms between different components in those architectures is
    % strongly affected by the algorithm nature.
    % (Esta sección debe incluir la descripción del problema resolver o idea a desarrollar, y la motivación para hacerlo. Es decir, cual es la importancia, innovación, aporte, y/o beneficio para la ciencia y/o la humanidad). Esta sección responde a la pregunta ¿Cuál es el problema que voy a resolver?
\subsection{Current works} 
TODO
% \emph{
%     (Esta sección debe incluir los enfoques usados actualmente para resolver el problema. Esta sección debe contener referencias bibliográficas a trabajos relacionados al proyecto.) Esta sección responde a la pregunta ¿Qué se ha hecho para resolver el problema?
% }
\subsection{Proposed solution}
% \emph{
    In order to accelerate a similarity search engine, we will develop a naive implementation of a 
    permutant-based metric space index without any major optimizations to study the feasibility of porting
    each subroutines involved based on their potential parallelism.

    After the study, we will port such algorithms as Intellectual Properties (IPs) to be validated and
    then implemented on the FGPA as accelerators for futher interconnect with the host-controller. Then a
    performance comparison will be executed in order to determine the feasibility of the solution.

    
    % (Esta sección debe incluir el planteamiento y justificación de la solución y/o idea, incluyendo aspectos novedosos.) Esta sección responde a la pregunta ¿Cómo voy a resolver el problema planteado?
% }

% \section{Hipótesis}
% % \emph{(En esta sección se deben incluir una lista de afirmaciones o suposiciones las cuales se esperar responder con el desarrollo del proyecto. La longitud máxima de esta sección es de 1/2 página.)}
% \begin{itemize}
%     \item{ Recurrent routines such as incremental sorting, median selection, permutant generation and comparison
%     are good targets to be accelerated on the FPGA.}
%     \item{ FPGA acceleration should improve the overall performance of metric spaces}
% % \item El uso de ... puede facilitar ....
% % \item El problema de .... puede estudiarse como ... 
% % \item Las técnicas usadas en ... pueden ser aplicables para resolver el problema de ...
% \end{itemize}



\section{Objectives}
% \emph{
%     % (En esta sección se deben especificar el objetivo general y los objetivos específicos del proyecto. Los objetivos deben reflejar lo que se espera lograr con el proyecto, evitando incluir características específicas de la solución. La longitud máxima de esta sección es de 1 página.)
%     }

\paragraph{Main objective}
% \emph{(Debe ser una sola frase que resuma lo que se quiere lograr.)} 
\begin{itemize}
\item Study the feasibility of implementing hardware-based accelerators for the Adapteva Parallella SoC.
\end{itemize}

\paragraph{Specific objectives}
% \emph{(Una lista de puntos que detallan el objetivo general.)}
\begin{itemize}
    \item Specify requirements and considerations to be accounted when porting general purpose algorithms to FPGAs.
    \item Study and implement a PL-PS data sharing sharing solution.
    \item Develop a functional FPGA-based hardware-accelerator prototype for a subset of routines involved on approximate similarity search.
    \item Deliver a solid guide to serve as a starting point to future computer scientist with little or no knowledge about hardware design. 
    % \item Design a hardware accelerator on the FPGA
    % \item Evaluate the performance impact for the given solution.
    % \item Deliver an effective guide to port algorithms to FPGAs
\end{itemize}



\section{Scope}
% \emph{(En esta sección se debe incluir una lista de puntos que definen los límites del trabajo. La longitud máxima de esta sección es de 1/2 página.)}
\begin{itemize}
    \item During this work we will not create a framework to develop new algorithms on FPGAs. 
    \item Also, we will not work on optimizations for the original versions of the tested routines.
    \item This work is limited only to research about FPGA hardware design, and to compare and contrast both 
        implementations.
\end{itemize}



\section{Methodology}
% \emph{(En esta sección se deben describir y justificar los métodos que se usarán para lograr las metas propuestas. Una buena opción es que, por cada objetivo, se describan una serie de pasos o métodos que se usarán para alcanzarlo. La longitud máxima de esta sección es de 1 página.)}


\paragraph{Milestone 1:} ``Approximate search algorithms and indices"
\begin{itemize}
    \item Analise previously developed hardware-accelerators.
    \item Research about resource sharing methods and techniques for the proposed architecture.
    \item Implement a simple hardware-accelerator IP Core on the FPGA.
    \item Study possible problems which could arise when porting common algorithms on custom hardware.
    \item Research about approximate search indices for metric spaces.
    \item Research about workarounds for high-dimensionality metric space datasets.
    \item Implement a permutant-based index and query algorithm.
    \item Analise the behiavour of the implemented solution and identify potential targets for hardware-acceleration
\end{itemize}

\paragraph{Milestone 2:} ``Hardware-accelerated index implementation"
\begin{itemize}
    \item Research about hardware prototyping.
    \item Research about hardware-software interconnection techniques and techlologies applicable to the
    target platform. 
    \item Implement as IP Cores the selected code fragments.
    \item Implement an interconection protocol for resource sharing between the two platforms.
    \item Detect possible bottlenecks or other problems derived from the interconection between the Atrix-7 
    FPGA and the ARMv7 A9 processor.
    \item Implement a loadable bitstream for the Parallella board and design according kernel modules.
    \item Replace software solution with custom hardware solution.
    \item Benchmark hardware implemented solution and constrast it with software implementation.
\end{itemize}

\section{Work plan}
% \emph{(En esta sección se debe definir como organizar y planificar, en términos de etapas y tiempo, las actividades a desarrollar así como los resultados a obtener. Esta sección debe incluir una Carta Gantt, la cual define fechas de inicio y término. La longitud máxima de esta sección es de 1 página, sin considerar la Carta Gantt.)}

\paragraph{Milestone 1:} ``Approximate search algorithms and indices"
\begin{itemize}
    \item Analise previously developed hardware-accelerators.
    \item Research about resource sharing methods and techniques for the proposed architecture.
    \item Implement a simple hardware-accelerator IP Core on the FPGA.
    \item Study possible problems which could arise when porting common algorithms on custom hardware.
    \item Research about approximate search indices for metric spaces.
    \item Research about workarounds for high-dimensionality metric space datasets.
    \item Implement a permutant-based index and query algorithm.
    \item Analise the behiavour of the implemented solution and identify potential targets for hardware-acceleration
\end{itemize}
\paragraph{Milestone 2:} ``Hardware-accelerated index implementation"
% \begin{itemize}
% % \item Research about hardware prototyping.
% % \item Research about hardware-software interconnection techniques and techlologies applicable to the target platform. 
% % \item Implement as IP Cores the selected code fragments.
% % \item Implement an interconection protocol for resource sharing between the two platforms.
% % \item Detect possible bottlenecks or other problems derived from the interconection between the Atrix-7 FPGA and the ARMv7 A9 processor.
% % \item Implement a loadable bitstream for the Parallella board and design according kernel modules.
% % \item Replace software solution with custom hardware solution.
% % \item Benchmark hardware implemented solution and constrast it with software implementation.
% \end{itemize}

\bibliographystyle{plain}
\bibliography{referencias}


\end{document}